\documentclass[12pt,a4paper]{article}

% ─── PAKETLER ───────────────────────────────────────────────────────────────
\usepackage[turkish]{babel}
\usepackage[utf8]{inputenc}
\usepackage[T1]{fontenc}
\usepackage{geometry}
\usepackage{xcolor}
\usepackage{tcolorbox}
\usepackage{enumitem}
\usepackage{booktabs}
\usepackage{longtable}
\usepackage{array}
\usepackage{graphicx}
\usepackage{fancyhdr}
\usepackage{titlesec}
\usepackage{hyperref}
\usepackage{mdframed}
\usepackage{fontawesome5}
\usepackage{multicol}
\usepackage{tikz}
\usepackage{pgfplots}
\pgfplotsset{compat=1.18}

% ─── SAYFA DÜZENİ ───────────────────────────────────────────────────────────
\geometry{
  top=2.5cm, bottom=2.5cm,
  left=2.5cm, right=2.5cm,
  headheight=15pt
}

% ─── RENKLER ────────────────────────────────────────────────────────────────
\definecolor{uyapBlue}{RGB}{0,71,133}
\definecolor{uyapGold}{RGB}{212,160,23}
\definecolor{uyapGreen}{RGB}{39,119,64}
\definecolor{uyapRed}{RGB}{180,35,35}
\definecolor{uyapGray}{RGB}{245,246,248}
\definecolor{uyapDark}{RGB}{30,40,55}
\definecolor{uyapLight}{RGB}{230,240,255}
\definecolor{uyapOrange}{RGB}{210,100,20}

% ─── TCOLORBOX STILLERI ─────────────────────────────────────────────────────
\tcbuselibrary{skins,breakable,listings}

\newtcolorbox{tanim}[1][]{
  enhanced, breakable,
  colback=uyapLight, colframe=uyapBlue,
  fonttitle=\bfseries\color{white},
  coltitle=white, attach boxed title to top left={yshift=-3mm,xshift=5mm},
  boxed title style={colback=uyapBlue, sharp corners},
  title={\faBook\ Tanım},
  #1
}

\newtcolorbox{onemli}[1][]{
  enhanced, breakable,
  colback=yellow!10, colframe=uyapGold,
  fonttitle=\bfseries,
  title={\faExclamationTriangle\ Önemli Uyarı},
  #1
}

\newtcolorbox{dikkat}[1][]{
  enhanced, breakable,
  colback=red!8, colframe=uyapRed,
  fonttitle=\bfseries\color{white},
  coltitle=white,
  attach boxed title to top left={yshift=-3mm,xshift=5mm},
  boxed title style={colback=uyapRed},
  title={\faShieldAlt\ DİKKAT},
  #1
}

\newtcolorbox{bilgi}[1][]{
  enhanced, breakable,
  colback=green!8, colframe=uyapGreen,
  fonttitle=\bfseries\color{white},
  coltitle=white,
  attach boxed title to top left={yshift=-3mm,xshift=5mm},
  boxed title style={colback=uyapGreen},
  title={\faInfoCircle\ Bilgi Notu},
  #1
}

\newtcolorbox{adimlar}[1][]{
  enhanced, breakable,
  colback=uyapGray, colframe=uyapDark,
  fonttitle=\bfseries\color{white},
  coltitle=white,
  attach boxed title to top left={yshift=-3mm,xshift=5mm},
  boxed title style={colback=uyapDark},
  title={\faCogs\ İşlem Adımları},
  #1
}

\newtcolorbox{yasal}[1][]{
  enhanced, breakable,
  colback=orange!8, colframe=uyapOrange,
  fonttitle=\bfseries\color{white},
  coltitle=white,
  attach boxed title to top left={yshift=-3mm,xshift=5mm},
  boxed title style={colback=uyapOrange},
  title={\faGavel\ Yasal Dayanak},
  #1
}

% ─── BAŞLIK STİLLERİ ────────────────────────────────────────────────────────
\titleformat{\section}
  {\Large\bfseries\color{uyapBlue}}
  {\thesection.}{0.5em}{}
  [\color{uyapGold}\titlerule]

\titleformat{\subsection}
  {\large\bfseries\color{uyapDark}}
  {\thesubsection.}{0.5em}{}

\titleformat{\subsubsection}
  {\normalsize\bfseries\color{uyapBlue}}
  {\thesubsubsection.}{0.5em}{}

% ─── ÜSTBILGI / ALTBILGI ────────────────────────────────────────────────────
\pagestyle{fancy}
\fancyhf{}
\fancyhead[L]{\small\color{uyapBlue}\bfseries Temel UYAP Bilgisi}
\fancyhead[R]{\small\color{uyapGray!60!black}
              İnfaz ve Koruma Memurları Eğitimi}
\fancyfoot[C]{\small\thepage}
\fancyfoot[R]{\small\color{gray} Adalet Bakanlığı --- CTEGM}
\renewcommand{\headrulewidth}{1.5pt}
\renewcommand{\headrule}{\hbox to\headwidth{%
  \color{uyapBlue}\leaders\hrule height \headrulewidth\hfill}}

% ─── HİPERBAĞLANTI ──────────────────────────────────────────────────────────
\hypersetup{
  colorlinks=true,
  linkcolor=uyapBlue,
  urlcolor=uyapBlue,
  pdftitle={Temel UYAP Bilgisi},
  pdfauthor={CTEGM Eğitim Birimi}
}

% ─── BELGE BAŞLANGICI ───────────────────────────────────────────────────────
\begin{document}

% ─── KAPAK ──────────────────────────────────────────────────────────────────
\begin{titlepage}
\pagecolor{uyapBlue}
\color{white}
\centering
\vspace*{2cm}

{\Huge\bfseries T.C.}\\[0.3cm]
{\Large\bfseries ADALET BAKANLIĞI}\\[0.2cm]
{\large Ceza ve Tevkifevleri Genel Müdürlüğü}

\vspace{1.5cm}
\color{uyapGold}
{\rule{12cm}{2pt}}
\vspace{0.5cm}

\color{white}
{\fontsize{32}{40}\selectfont\bfseries TEMEL UYAP BİLGİSİ}

\vspace{0.5cm}
\color{uyapGold}
{\rule{12cm}{2pt}}

\vspace{1.5cm}
\color{white}
{\LARGE\bfseries BÖLÜM 9}

\vspace{1cm}
\begin{tcolorbox}[
  colback=white!15!uyapBlue,
  colframe=uyapGold,
  width=10cm,
  arc=8pt
]
\centering
\large
\textbf{Hedef Kitle:}\\
Ceza İnfaz Kurumlarında Sözleşmeli\\
Göreve Başlayan\\
\textbf{İnfaz ve Koruma Memurları}\\[0.5cm]
\textbf{Toplam Süre:} 5 Saat\\
(3 Saat Yüz Yüze + 2 Saat Asenkron)
\end{tcolorbox}

\vfill
{\large 2024 --- Ankara}
\end{titlepage}
\pagecolor{white}

% ─── İÇİNDEKİLER ────────────────────────────────────────────────────────────
\tableofcontents
\newpage

% ════════════════════════════════════════════════════════════════════════════
\section{Temel UYAP Bilgisi}
% ════════════════════════════════════════════════════════════════════════════

\subsection{UYAP'ın Tanımı ve Amaçları}

\begin{tanim}
\textbf{UYAP (Ulusal Yargı Ağı Projesi)}, adalet hizmetlerinin en kısa
zamanda, en az maliyetle, \textit{şeffaf, etkili, verimli, hızlı, güvenilir,
tarafsız ve denetlenebilir} şekilde yerine getirilmesini temin etmek üzere
geliştirilen ve Türkiye'deki e-devlet uygulamalarının bir parçasını
oluşturan \textbf{elektronik adalet sistemidir}.
\end{tanim}

\vspace{0.5cm}

UYAP, Adalet Bakanlığı Bilgi İşlem Dairesi Başkanlığı tarafından hayata
geçirilmiş \textbf{merkezi bir bilişim sistemi} olarak aşağıdaki tüm birimleri
kapsamaktadır:

\begin{multicols}{2}
\begin{itemize}[leftmargin=1.5em]
  \item[\faGavel] Tüm mahkemeler
  \item[\faSearch] Savcılık birimleri
  \item[\faLock] Cezaevleri
  \item[\faBalanceScale] Diğer yargı birimleri
  \item[\faBuilding] Adalet Bakanlığı merkez teşkilâtı
  \item[\faLink] Bağlı ve ilgili birimler
  \item[\faUniversity] İlgili devlet kurumları
  \item[\faNetworkWired] Entegre sistemler
\end{itemize}
\end{multicols}

\subsubsection{Entegre Sistemler}

\begin{center}
\begin{tabular}{lll}
\toprule
\textbf{Sistem} & \textbf{Kurum} & \textbf{Sağlanan Bilgi} \\
\midrule
MERNİS   & İçişleri Bakanlığı  & Nüfus / kimlik kaydı \\
Adli Sicil & Adalet Bakanlığı  & Sabıka kaydı \\
TAKBİS   & Tapu ve Kadastro    & Tapu kaydı \\
POLNET   & Emniyet Genel Müd.  & Aranan kişi sorgulama \\
Merkez Bankası & TCMB            & Döviz kuru bilgisi \\
\bottomrule
\end{tabular}
\end{center}

% ─────────────────────────────────────────────────────────────────────────────
\subsection{UYAP'ın Faydaları}
% ─────────────────────────────────────────────────────────────────────────────

UYAP'ın yargı sistemine sağladığı başlıca katkılar şu başlıklar altında
incelenebilir:

\subsubsection{Hız ve Verimlilik}
\begin{itemize}
  \item Yargı birimleri arasında yazışmalar \textbf{kâğıtsız ofis} mantığında
        elektronik ortamda yapılmaktadır.
  \item Davalardaki talimat işlemleri UYAP'ta \textbf{online} olarak karşı
        birime iletilmekte, cevap anında gönderilmektedir.
  \item Temyiz edilen dosyaların Yargıtay'a gönderilmesi ve geri dönmesi
        \textbf{elektronik ortamda anlık} sağlanmaktadır.
  \item Mahkemelerin yılsonu devir işlemleri \textbf{çok kısa sürede}
        tamamlanmaktadır.
\end{itemize}

\subsubsection{Doğruluk ve Tutarlılık}
\begin{itemize}
  \item Veriler \textbf{tek yerde} tutulduğundan eksiksiz, doğru ve
        güncel bilgiye anında ulaşılabilmektedir.
  \item Tevzi işlemleri \textbf{otomatik ve adil} şekilde yapılmaktadır;
        müdahale ve yolsuzlukların önüne geçilmektedir.
  \item Mükerrer davaların tespiti mümkün hale gelmektedir.
\end{itemize}

\subsubsection{Denetim ve Şeffaflık}
\begin{itemize}
  \item Doküman Yönetim Sistemi sayesinde \textbf{kimin, ne zaman,
        ne yaptığı} takip edilebilmektedir.
  \item Kasa hesapları sistem tarafından \textbf{otomatik kontrol}
        edilmekte; usulsüzlükler anında ortaya çıkarılmaktadır.
  \item Merkezden taşradaki adliyelerin \textbf{online denetimi}
        mümkün hale gelmiştir.
\end{itemize}

\subsubsection{Ceza İnfaz Kurumlarına Özel Faydalar}

\begin{bilgi}
\begin{itemize}
  \item Mahkûm ve tutuklu sayıları \textbf{anlık raporlanabilmektedir}.
  \item Kadın/erkek/çocuk ayrımı \textbf{merkezden izlenebilmektedir}.
  \item Suç türlerine göre istatistikler \textbf{otomatik üretilmektedir}.
  \item Ceza infaz kurumları \textbf{etkin biçimde takip} edilebilmektedir.
  \item İnfaz süresi, koşullu salıverilme tarihi gibi hesaplamalar
        \textbf{sistem yardımıyla} yapılmaktadır.
\end{itemize}
\end{bilgi}

% ─────────────────────────────────────────────────────────────────────────────
\subsection{UYAP'a Veri Girişi Sırasında Dikkat Edilmesi Gereken Hususlar}
% ─────────────────────────────────────────────────────────────────────────────

\begin{onemli}
Mevzuat hükümlerine göre UYAP'ın kullanılması \textbf{zorunlu} olup bütün
yazışmalar, iş ve işlemler UYAP aracılığıyla yürütülmektedir. Görevli ve
yetkili personel UYAP'a verileri \textbf{doğru, eksiksiz ve zamanında}
girmekle yükümlüdür.
\end{onemli}

\subsubsection{Hatalı Veri Girişinin Sonuçları}

\begin{center}
\begin{tabular}{lp{9cm}}
\toprule
\textbf{Etkilenen Alan} & \textbf{Sonuç} \\
\midrule
Adli istatistik & Doğru üretilememesi; yanlış politika geliştirilmesi \\
Kişi hakları    & Hak kayıpları, mağduriyetler \\
Yargılama süreci & Süreçlerin uzaması \\
Denetim          & Teftiş ve denetimlerin yapılamaması \\
Kurum bütçesi    & Maliyetlerin artması \\
Personel         & İdarî/adlî soruşturma açılması \\
\bottomrule
\end{tabular}
\end{center}

\subsubsection{Veri Giriş Sorunlarının Kaynakları}

\begin{itemize}
  \item[\faTimesCircle] Yeterli bilgi, beceri ve yetkinlik eksikliği
  \item[\faTimesCircle] İş yoğunluğu
  \item[\faTimesCircle] İş bölümünün bulunmaması
  \item[\faTimesCircle] Dikkat eksikliği
  \item[\faTimesCircle] Veri girişinin önemi konusunda farkındalık eksikliği
\end{itemize}

\subsubsection{Gerekli Şifreleri Oluşturma ve Kullanma}

UYAP uygulamalarını kullanabilmek için dört temel parolaya ihtiyaç
duyulmaktadır:

\vspace{0.3cm}

\begin{center}
\begin{tabular}{p{3.5cm}p{5cm}p{5cm}}
\toprule
\textbf{Parola Türü} & \textbf{Nasıl Alınır?} &
\textbf{Nasıl Değiştirilir?} \\
\midrule
Portal Parolası &
  Parola Talep Formu $\to$ Faks $\to$ Bilgi İşlem Daire Bşk. &
  UYAP Portal $\to$ Şifre İşlemleri \\[4pt]
E-Posta Parolası &
  Portal parolasıyla giriş $\to$ Şifre İşlemleri &
  Şifre İşlemleri sekmesi \\[4pt]
Domain (Bilgisayar) Parolası &
  sifredegistirme.uyap.gov.tr $\to$ E-posta onay kodu &
  CTRL+ALT+DELETE $\to$ Parola Değiştir \\[4pt]
E-İmza Parolası &
  TÜBİTAK KamuSM $\to$ kamusm.gov.tr &
  KamuSM sistemi üzerinden \\
\bottomrule
\end{tabular}
\end{center}

\begin{adimlar}[title={\faCogs\ Portal Parolası Alma Adımları}]
\begin{enumerate}
  \item Parola Talep Formu doldurulur
  \item Ekran çıktısı alınır
  \item Personel \textbf{ve} kurum amiri imzalar
  \item Forma yazılı numaraya \textbf{faks çekilir}
  \item Bilgi İşlem Dairesi Başkanlığı parolayı oluşturur
  \item Personele \textbf{telefonla} verilir
  \item Portal $\to$ Şifre İşlemleri $\to$ Parola değiştirilir $\checkmark$
\end{enumerate}
\end{adimlar}

\begin{adimlar}[title={\faCogs\ E-İmza Parolası Alma Adımları}]
\begin{enumerate}
  \item \texttt{kamusm.gov.tr} $\to$ Bireysel İşlemler $\to$
        Şifreli Giriş
  \item TC Kimlik No + Güvenlik Sözcüğü girişi
  \item Sisteme kayıtlı telefon numarası doğrulanır
  \item ONAY $\to$ Cep telefonuna \textbf{SMS onay kodu} gelir
  \item Onay kodu girilir $\to$ Sisteme giriş
  \item E-imza parolası oluşturulur $\checkmark$
\end{enumerate}
\end{adimlar}

\begin{dikkat}
\begin{itemize}
  \item Şifrenizi \textbf{kimseyle paylaşmayın.}
        (53/1 sayılı Genelge Md.\ 8: Şifreyi kullanan personel sorumludur.)
  \item Sistemden \textbf{çıkış yapmadan} bilgisayarı bırakmayın.
  \item Her işlem \textbf{kullanıcı adınıza} kayıt altına alınır.
  \item Yanlış işlem yapan personel ve
        \textbf{amir birlikte sorumlu tutulur.}
\end{itemize}
\end{dikkat}

% ════════════════════════════════════════════════════════════════════════════
\section{UYAP Doküman Editörü}
% ════════════════════════════════════════════════════════════════════════════

\begin{tanim}
\textbf{UYAP Doküman Editörü}, UYAP sisteminde kullanılan,
e-imza ile belge imzalama özelliğine sahip, \texttt{.udf} uzantılı dosyaları
destekleyen \textbf{kelime işlemci uygulamasıdır}.
\end{tanim}

\subsection{Doküman Editörünün Kurulumu}

\begin{adimlar}
\begin{enumerate}
  \item \textbf{Java Kurulumu:}
        \texttt{java.com/tr} $\to$ Java İndir (32 bit) $\to$ Kur
  \item \textbf{E-İmza Kart Okuyucu Sürücüleri:}
        Sertifika hizmet sağlayıcısından alınan sürücüler yüklenir
  \item \textbf{UYAP Doküman Editörü:}
        \texttt{uyap.gov.tr} $\to$ UYAP Kelime İşlemci İndir
        $\to$ ZIP çıkart $\to$ Kur $\to$ İleri $\to$ Tamamla $\checkmark$
\end{enumerate}
\end{adimlar}

\subsection{Belge Kaydetme}

\textbf{İlk Kaydetme:}
\begin{quote}
Dosya $\to$ Farklı Kaydet $\to$ Konum Seç $\to$ Kaydet
\end{quote}

\textbf{Çalışırken Kaydetme:}
\begin{quote}
Dosya $\to$ Kaydet \quad \textit{veya} \quad
Hızlı Erişim Çubuğu $\to$ Kaydet simgesi
\end{quote}

\begin{bilgi}
Editör varsayılan olarak \texttt{\textbf{.udf}} biçiminde kaydeder.
Farklı biçimler: \texttt{.usf}, \texttt{.doc}, \texttt{.pdf},
\texttt{.odt}, \texttt{.rtf}, \texttt{.tif}, \texttt{.jpg}
\end{bilgi}

\subsection{Doküman İmzalama}

\begin{adimlar}[title={\faCogs\ E-İmza ile İmzalama Adımları}]
\begin{enumerate}
  \item E-imzayı bilgisayara tak
  \item Araçlar Menüsü $\to$ İmza Kütüphanesi $\to$ Firma kartını seç
  \item \textbf{İmzala} simgesine tıkla
  \item E-imza şifresini gir
  \item Tamam'a bas
  \item Sağ üst köşede \textbf{kurdele} görünür
        (\textit{Kurdele = İmza doğrulandı}) $\checkmark$
\end{enumerate}
\end{adimlar}

\begin{onemli}
İmzalı evrakta değişiklik yapılırsa \textbf{imza bilgileri kaybolur!}
Değişiklik sonrası evrak \textbf{yeniden e-imza ile imzalanmalıdır.}
\end{onemli}

\subsection{Tümünü Farklı Kaydet}

\begin{adimlar}
\begin{enumerate}
  \item Dosya $\to$ Tümünü Farklı Kaydet
  \item Evrak veya klasör seç (\texttt{.udf} dosyaları)
  \item Çıktı klasörünü belirle
  \item Format seç: \texttt{doc / odt / pdf / rtf}
  \item Dönüştür $\checkmark$
\end{enumerate}
\end{adimlar}

\begin{bilgi}
\textbf{Not:} C: sürücüsünde LibreOffice veya OpenOffice
(v3 önerilir) kurulu olması gerekmektedir.
\end{bilgi}

\subsection{Belge Yazdırma}

\begin{adimlar}
\begin{enumerate}
  \item Dosya $\to$ Yazdır
  \item Kopya sayısını belirle
  \item Doğru yazıcı seçili mi kontrol et
  \item Sayfa düzenini ayarla
  \item Yazdır $\checkmark$
\end{enumerate}
\end{adimlar}

\begin{bilgi}
\textbf{İpucu:} Evrakta Web ID (evrakın aslının kontrol edilmesini
sağlayan kod) bulunması durumunda \textit{Metni Sayfaya Sığdır}
seçeneği aktif hale gelir. Web ID'siz yazdırmak için bu seçeneği
pasife alın.
\end{bilgi}

\subsection{Görev Çubuğu ve Kişisel Menü}

\begin{adimlar}[title={\faCogs\ Kişiselleştirme}]
\begin{enumerate}
  \item Herhangi bir ikona \textbf{sağ tık} yapın
  \item \textit{``Araç Çubuğuna Ekle''} $\to$ Görev Çubuğuna ekler
  \item \textit{``Kişisel Menüye Ekle''} $\to$ Kişisel Menüye ekler
  \item Kaldırmak için: Çubuk/menüdeki ikona sağ tık $\to$ Kaldır
\end{enumerate}
\end{adimlar}

\begin{center}
\begin{tabular}{ll}
\toprule
\textbf{Kısayol} & \textbf{İşlev} \\
\midrule
Alt veya F10    & Menülere hızlı erişim \\
Ctrl + F1       & Menüyü gizle/göster \\
Çift tıkla      & Menüyü gizle \\
\bottomrule
\end{tabular}
\end{center}

% ════════════════════════════════════════════════════════════════════════════
\section{Personel Uygulamaları (Ortak İşlemler)}
% ════════════════════════════════════════════════════════════════════════════

\begin{adimlar}[title={\faCogs\ Sisteme Erişim}]
UYAP Portal $\to$ Kullanıcı Bilgileri Gir $\to$
Personel Uygulamaları Rolü $\to$ Aktif Görevlendirmeler $\to$
Ortak İşlemler
\end{adimlar}

\subsection{Özlük İşlemleri}

\subsubsection{Aile Fert Bilgileri}

\begin{quote}
Personel İşlemleri $\to$ Özlük Bilgisi $\to$
Personel Bilgileri $\to$ Aile Fert Bilgileri
\end{quote}

\begin{onemli}
NAS (Atanma Talep Formu)'nda \textbf{Eş Durumu} tayin gerekçesi seçilecekse
eşin çalışma ve görev yeri bilgileri \textbf{mutlaka güncel} olmalıdır!
\end{onemli}

\begin{dikkat}
Kaydet simgesine tıklamadan kaydetme işlemi \textbf{gerçekleşmez.}
Kaydetmeden ekranı kapattığınızda bilgiler eski formunda kalır.
\end{dikkat}

\subsection{İzin Bilgileri}

\begin{adimlar}[title={\faCogs\ İzin Talep Akışı}]
\begin{enumerate}
  \item Personel İşlemleri $\to$ Personel İzin Talep Formu $\to$
        \textbf{Yeni Talep}
  \item İzin türü seç
  \item İzne ayrılış ve bitiş tarihlerini gir
        (İzin süresi \textbf{otomatik hesaplanır})
  \item İzinde geçirilecek adres, il ve ilçe bilgilerini gir
  \item Gerekli evrakları ekle
  \item Giden evrak: Konu + Açıklama yaz
        (Açıklama: Hangi yıla ait? Örn: 2024/3)
  \item Sıralı amirleri seç $\to$ Kapat
  \item Kaydet $\to$ Evrak Göster + İmzala Onayla aktif hale gelir
  \item İmzala/Onayla $\to$ E-imza şifresi $\to$ Tamam
  \item Onay sürecine gönderildi $\checkmark$
\end{enumerate}
\end{adimlar}

\subsection{Personel Atanma Talep Formu (NAS)}

\begin{center}
\begin{tabular}{p{4cm}p{9cm}}
\toprule
\textbf{Alan} & \textbf{Değer} \\
\midrule
Zorunlu atamaya tabi mi? & \textbf{HAYIR} \\
Talep Türü & \textbf{GENEL} \\
Talep Nedeni & \textbf{DEĞİŞİKLİK TALEBİ} \\
Tayin Gerekçesi & \textbf{TALEP} \\
Maksimum tercih & \textbf{5 kurum} \\
\bottomrule
\end{tabular}
\end{center}

\begin{bilgi}
Sıralama değişikliği için \textbf{Yukarı/Aşağı} butonlarını kullanın.
Tercih silmek için \textbf{Sil} butonunu kullanın.
Telefon ve e-posta bilgileri eksiksiz doldurulmalıdır.
\end{bilgi}

\subsection{Emlak ve Servet Bilgileri (Mal Bildirimi)}

\subsubsection{Beyan Zorunluluğu}

\begin{center}
\begin{tabular}{p{5cm}p{8cm}}
\toprule
\textbf{Durum} & \textbf{Beyan Türü} \\
\midrule
Göreve ilk girişte & Genel Beyan \\
Sonu 0 ve 5 ile biten yıllar & Genel Beyan \\
Mal varlığında değişiklik & Ek Beyan (30 iş günü içinde) \\
\bottomrule
\end{tabular}
\end{center}

\begin{yasal}
\textbf{657 sayılı DMK Madde 125:} Belirlenen durum ve sürelerde mal
bildiriminde bulunmayan memurlar için \textbf{kademe ilerlemesinin
durdurulması} cezası öngörülmüştür.
\end{yasal}

\begin{bilgi}
\begin{itemize}
  \item Memur \textbf{kendisi, eşi} ve \textbf{velayetindeki çocukları}
        için bildirim yapar.
  \item Aynı mal grubunda aynı isim kullanılamaz
        (Otomobil 1, Otomobil 2 şeklinde girilmeli).
  \item E-imza yoksa: Önlü arkalı yazdır $\to$ Islak imzala $\to$
        Kapalı zarf $\to$ İlgili birime teslim.
\end{itemize}
\end{bilgi}

% ════════════════════════════════════════════════════════════════════════════
\section{UYAP Mevzuat Programı}
% ════════════════════════════════════════════════════════════════════════════

\subsection{Programın İndirilmesi ve Kurulması}

\begin{adimlar}
\begin{enumerate}
  \item \texttt{mevzuat.adalet.gov.tr} adresine git
  \item ``Mevzuat programını indirmek için tıklayınız'' bağlantısını tıkla
  \item Güvenlik kodunu gir $\to$ İndir
  \item Kurulum paketini çalıştır
        (\textit{Yönetici izni gerekmez; her kullanıcı yapabilir})
  \item Program her açılışta \textbf{otomatik güncellenir} $\checkmark$
\end{enumerate}
\end{adimlar}

\subsection{Mevzuat Programı ile Yapılabilecekler}

\begin{multicols}{2}
\textbf{Mevzuat Arama Kriterleri:}
\begin{itemize}
  \item Mevzuat Türü
  \item Mevzuat Adı
  \item Mevzuat Numarası
  \item Mevzuat içeriği
  \item Resmî Gazete Tarih ve Sayısı
\end{itemize}

\columnbreak

\textbf{İçtihat Arama Kriterleri:}
\begin{itemize}
  \item İçtihat Türü
  \item Metin İçeriği
  \item Daire/Kurul Adı
  \item Esas Numarası
  \item Karar Numarası
  \item Aranacak Kavram
\end{itemize}
\end{multicols}

\begin{center}
\begin{tabular}{lp{9cm}}
\toprule
\textbf{Erişilebilecek İçerik} & \textbf{Kapsam} \\
\midrule
Güncel mevzuat       & Her gün güncellenen tüm mevzuat \\
Yüksek yargı içtihadı & 38.000'i aşan karar \\
Anayasa Mahkemesi    & Tüm kararlar \\
AİHM                 & Türkiye ile ilgili kararlar \\
Yargıtay / Danıştay  & Emsal kararlar \\
Uyuşmazlık Mahkemesi & Tüm kararlar \\
Adalet Dergisi       & Yayımlanan makaleler \\
\bottomrule
\end{tabular}
\end{center}

% ════════════════════════════════════════════════════════════════════════════
\section{Ölçme ve Değerlendirme}
% ════════════════════════════════════════════════════════════════════════════

\begin{center}
\large\bfseries
10 Soruluk Değerlendirme Testi\\[0.3cm]
\normalsize (Her soru 10 puan --- Toplam 100 puan)
\end{center}

\begin{enumerate}
  \item UYAP'ın temel amacı aşağıdakilerden hangisidir?\\
    A) Adalet hizmetlerini hızlı, güvenilir ve etkin yürütmek \quad
    B) Mahkemelerde dosya sayısını azaltmak\\
    C) İnternet hizmetlerini yaygınlaştırmak \quad
    D) Sadece vatandaşların dosya sorgulamasını sağlamak\\
    E) Evrak taramasını zorunlu hale getirmek

  \item UYAP'ta veri girişinin doğru, eksiksiz ve zamanında
        yapılmamasının sonucu aşağıdakilerden hangisidir?\\
    A) İnternet kesintilerinin artması \quad
    B) Adli istatistiklerin doğru üretilememesi\\
    C) Evrakların otomatik imzalanması \quad
    D) Personelin izinlerinin iptal edilmesi\\
    E) Yeni mevzuatın yayımlanamaması

  \item E-imza parolası hangi kurum tarafından üretilir?\\
    A) Adalet Bakanlığı \quad
    B) UYAP Teknik Ofis Yetkilisi\\
    C) TÜBİTAK Kamu Sertifikasyon Merkezi \quad
    D) Yargıtay Bilgi İşlem\\
    E) Maliye Bakanlığı

  \item UYAP Doküman Editörü'nün belgeyi varsayılan olarak kaydettiği
        dosya uzantısı nedir?\\
    A) .doc \quad B) .udf \quad C) .pdf \quad D) .rtf \quad E) .odt

  \item Aşağıdakilerden hangisi UYAP'ın sağladığı faydalardan biri
        \textbf{değildir}?\\
    A) Yargı birimleri arasında hızlı elektronik yazışma \quad
    B) Veri bütünlüğünün sağlanması\\
    C) Temyiz dosyalarının elektronik gönderilmesi \quad
    D) Mahkeme kararlarının otomatik hazırlanması\\
    E) Online talimat işlemlerinin yapılması

  \item Doküman Editöründe imzalı evrakta değişiklik yapılırsa ne olur?\\
    A) Sistem değişikliği kabul etmez. \quad
    B) Eski imza geçerli kalır.\\
    C) Değişiklik sonrası evrakın tekrar imzalanması gerekir.\\
    D) Evrak kendiliğinden iptal olur. \quad
    E) Evrak otomatik PDF'e dönüşür.

  \item UYAP Personel Uygulamaları menüsünden yapılabilecek işlemlerden
        biri aşağıdakilerden hangisidir?\\
    A) Mahkeme tensip tutanağı hazırlamak \quad
    B) TAKBİS kayıtlarını düzenlemek\\
    C) Disiplin cezası vermek \quad
    D) Elektronik imza sertifikası üretmek\\
    E) Aile fert bilgilerini güncellemek

  \item İzin Talep Formunda izin süresi nasıl hesaplanır?\\
    A) Kullanıcı manuel hesaplar.\\
    B) Sadece izin amiri onaylayınca hesaplanır.\\
    C) Sistem ayrılış ve dönüş tarihine göre otomatik hesaplar.\\
    D) MEB izin çizelgesine göre hesaplanır.\\
    E) Personel Dairesi hesaplar.

  \item UYAP Mevzuat Programı aşağıdakilerden hangisini sağlar?\\
    A) Tutuklu bilgilerinin kaydını sağlar.\\
    B) Günlük güncellenen mevzuata erişimi sağlar.\\
    C) Personel izin sürelerinin hesabını yapar.\\
    D) Elektronik imza doğrulamasını yapar.\\
    E) Cezaevi yemek listelerini düzenler.

  \item UYAP'a girilen veri hangi hukuki niteliğe sahiptir?\\
    A) Tavsiye niteliği taşır. \quad
    B) Tutanak niteliğindedir.\\
    C) Belge değeri yoktur. \quad
    D) Bilgi formu yerine geçer.\\
    E) Resmî yazı hükmündedir.
\end{enumerate}

\vspace{1cm}
\begin{center}
\begin{tcolorbox}[colback=uyapBlue!10,colframe=uyapBlue,
                  title={\color{white}\bfseries Cevap Anahtarı},
                  width=12cm]
\centering\large
\begin{tabular}{cccccccccc}
1 & 2 & 3 & 4 & 5 & 6 & 7 & 8 & 9 & 10 \\
\hline
\textbf{A} & \textbf{B} & \textbf{C} & \textbf{B} & \textbf{D} &
\textbf{C} & \textbf{E} & \textbf{C} & \textbf{B} & \textbf{E}
\end{tabular}
\end{tcolorbox}
\end{center}

% ─── KAYNAKÇA ───────────────────────────────────────────────────────────────
\newpage
\section*{Kaynakça}
\addcontentsline{toc}{section}{Kaynakça}

\begin{itemize}
  \item Adalet Bakanlığı CTEGM --- İnfaz ve Koruma Memurları Temel
        Eğitim Kitabı, Bölüm 9 (s.\ 347--424)
  \item 53/1 Sayılı UYAP Genelgesi, 30/03/2007
  \item \url{http://www.uyap.gov.tr}
  \item \url{http://www.kamusm.gov.tr}
  \item \url{http://www.mevzuat.adalet.gov.tr}
  \item 657 Sayılı Devlet Memurları Kanunu
  \item 5275 Sayılı Ceza ve Güvenlik Tedbirlerinin
        İnfazı Hakkında Kanun
  \item 6698 Sayılı Kişisel Verilerin Korunması Kanunu (KVKK)
\end{itemize}

\end{document}